\documentclass[twoside]{scrartcl}
\usepackage[frame]{xy}
\usepackage{tabularx}
\usepackage[latin1]{inputenc}
\setlength{\voffset}{0.5cm}
\setlength{\hoffset}{-2.0cm}
\setlength{\topmargin}{0cm}
\setlength{\headheight}{0.5cm}
\setlength{\headsep}{1cm}
\setlength{\topskip}{0pt}
\setlength{\oddsidemargin}{1.0cm}
\setlength{\evensidemargin}{1.0cm}
\setlength{\textwidth}{19.2cm}
\setlength{\textheight}{24.5cm}
\setlength{\footskip}{1cm}
\setlength{\parindent}{0pt}
\renewcommand{\baselinestretch}{1}
\begin{document}

\newlength{\descrwidth}\setlength{\descrwidth}{10cm}

\newsavebox{\hdr}
\sbox{\hdr}{
  \fontfamily{cmss}\fontsize{10pt}{12pt}\selectfont

  \parbox{\textwidth}{
    \parbox[b]{12cm}{
      <%company%>
      
      <%address%>}\hfill
    \begin{tabular}[b]{rr@{}}
    Telefon & <%tel%>\\
    Telefax & <%fax%>
    \end{tabular}

    \rule[1.5ex]{\textwidth}{0.5pt}
  }
}
    
\fontfamily{cmss}\fontshape{n}\selectfont

\markboth{<%company%>\hfill <%invnumber%>}{\usebox{\hdr}}

\pagestyle{myheadings}
%\thispagestyle{empty}     use this with letterhead paper

<%pagebreak 90 27 37%>
\end{tabular*}

  \rule{\textwidth}{2pt}
  
  \hfill
  \begin{tabularx}{7cm}{Xr@{}}
  \textbf{Subtotal} & \textbf{<%sumcarriedforward%>} \\
  \end{tabularx}

\newpage

\markright{<%company%>\hfill <%invnumber%>}

\vspace*{-12pt}

\begin{tabular*}{\textwidth}{@{}lp{\descrwidth}@{\extracolsep\fill}rlrrr@{}}
  \textbf{Nummer} & \textbf{Artikel} & \textbf{Anz} &
    \textbf{Einh} & \textbf{Preis} & \textbf{Rab} & \textbf{Total} \\
  & �bertrag von Seite <%lastpage%> & & & & & <%sumcarriedforward%> \\
<%end pagebreak%>


\fontfamily{cmss}\fontsize{10pt}{12pt}\selectfont

\vspace*{2cm}

<%name%>

<%address1%>

<%if address2%>
<%address2%>
<%end address2%>

<%city%> <%state%> <%zipcode%>

<%if country%>
<%country%>
<%end country%>

\vspace{3.5cm}

\textbf{R E C H N U N G}
\hfill
\begin{tabular}[t]{l@{\hspace{0.3cm}}l}
  \textbf{Datum} & <%invdate%> \\
  \textbf{Nummer} & <%invnumber%> \\
\end{tabular}

\vspace{1cm}

\begin{tabular*}{\textwidth}{@{}lp{\descrwidth}@{\extracolsep\fill}rlrrr@{}}
  \textbf{Nummer} & \textbf{Artikel} & \textbf{Anz} &
    \textbf{Einh} & \textbf{Preis} & \textbf{Rab} & \textbf{Total} \\
<%foreach number%>
  <%number%> & <%description%> & <%qty%> &
    <%unit%> & <%sellprice%> & <%discount%> & <%linetotal%> \\
<%end number%>
\end{tabular*}


\parbox{\textwidth}{
\rule{\textwidth}{2pt}

\vspace{0.2cm}

\hfill
\begin{tabularx}{7cm}{Xr@{}}
  \textbf{Zwischensumme} & \textbf{<%subtotal%>} \\
<%foreach tax%>
  <%taxdescription%> auf <%taxbase%> & <%tax%> \\
<%end tax%>
  \hline
  \textbf{Total} & \textbf{<%invtotal%>} \\
<%if paid%>
  \textbf{Bezahlt} & <%paid%> \\
<%end paid%>
<%if total%>
  \textbf{Bezahlbar} & \textbf{<%total%>} \\
<%end total%>
\end{tabularx}

\vspace{0.3cm}

\hfill
  Alle Preise in \textbf{<%currency%>}.

\vspace{12pt}

<%if notes%>
  <%notes%>
<%end if%>

}

%\vfill
%\centerline{\textbf{salute}}

\renewcommand{\thefootnote}{\fnsymbol{footnote}}

\footnotetext[1]{\tiny
Rechnung ist bezahlbar innerhalb von <%terms%> Tagen.
Nach dem <%duedate%> werden Zinsen zu einem
monatlichen Satz von 1.5\% verrechnet.
Waren bleiben im Besitz von <%company%> bis die Rechnung voll bezahlt ist.
R�ckgaben werden mit 10 Prozent Lagergeb�hren belastet. Besch�digte Waren
und Waren ohne eine R�ckgabenummer werden nicht entgegengenommen.
}

\end{document}


